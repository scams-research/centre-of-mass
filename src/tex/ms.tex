% Define document class
\documentclass[reprint,superscriptaddress,nobibnotes,amsmath,amssymb,aip]{revtex4-2}
\usepackage{showyourwork}
\usepackage[version=4]{mhchem}
\usepackage{graphicx}
\usepackage[utf8]{inputenc}
\graphicspath{{figures/}}
\usepackage{bm}
\usepackage{setspace}
\usepackage{siunitx}
\sisetup{list-units=bracket,
         list-final-separator={, and },
         range-units=bracket,
         range-phrase={, },
         uncertainty-mode=separate
         }
\usepackage{lipsum}
\newcommand{\papertitle}{On the Estimation of Centre of Mass in Periodic Systems}
\usepackage[noabbrev,nameinlink,capitalize]{cleveref}
\crefname{equation}{Eq.}{Eqs.}
\crefformat{equation}{Eq.~#2#1#3}
\crefrangeformat{equation}{Eqs.~#3#1#4 to~#5#2#6}
\crefmultiformat{equation}{Eqs.~#2#1#3}{ and~#2#1#3}{, #2#1#3}{ and~#2#1#3}
\crefname{figure}{Fig.}{Figs.}

%% Apr 2021: AIP requests that the corresponding 
%% email to be moved after the affiliations
\makeatletter
\def\@email#1#2{%
 \endgroup
 \patchcmd{\titleblock@produce}
  {\frontmatter@RRAPformat}
  {\frontmatter@RRAPformat{\produce@RRAP{*#1\href{mailto:#2}{#2}}}\frontmatter@RRAPformat}
  {}{}
}%
\makeatother

\renewcommand{\refname}{References}

\begin{document}

\let\oldaddcontentsline\addcontentsline
\renewcommand{\addcontentsline}[3]{}

\title[Accurate Centre of Mass Estimation]{\papertitle}

\author{Harry Richardson}
  \affiliation{Centre for Computational Chemistry, School of Chemistry, University of Bristol, Cantock's Close, Bristol, BS8 1TS, UK.}
\author{Josh Dunn}
  \affiliation{Centre for Computational Chemistry, School of Chemistry, University of Bristol, Cantock's Close, Bristol, BS8 1TS, UK.}
\author{Andrew R. McCluskey}
\email{andrew.mccluskey@bristol.ac.uk}
  \affiliation{Centre for Computational Chemistry, School of Chemistry, University of Bristol, Cantock's Close, Bristol, BS8 1TS, UK.}
  \affiliation{Diamond Light Source, Harwell Campus, Didcot, OX11 0DE, UK.}

\date{\today}

\begin{abstract}
Calculation of the centre of mass of some species from a periodic cell is an important aspect of chemical and physical simulation. 
One popular approach, described by Bai and Breen, calculates the centre of mass via the projection of the individual particles' coordinates onto a circle.
This approach suffers from some numerical error as it is mathematically equivalent to finding the first moment of the Fourier series of the mass density. 
Here we discuss this inaccuracy and propose an extension that overcomes it, enabling improved accuracy across computational simulation. 
\end{abstract}

\maketitle

\section{Introduction}
\label{sec:intro}

The calculation of the centre of mass of a group of particles, such as in a molecule, is important across computational chemistry and physics.\cite{zhang_chemically_2024,happel_coordinated_2024,maggi_universality_2021,grillo_molecular_2023,bullerjahn_unwrapping_2023,jaeger-honz_systematic_2024}
However, computing the centre of mass is non-trivial for systems with periodic boundary conditions. 
Any group of particles will have the same number of valid centres of mass---positions in the simulation cell where the weighted relative positions of the masses sum to zero---as the number of particles (\cref{fig:problems}a). 
However, in the physical sciences, we may be concerned with the centre of mass of a single bonded molecule or a defined group of particles. 
For a single bonded molecule, there is only one correct answer for the centre of mass.
Practically speaking, for most simulations, it is most desirable to compute the centre of mass that minimises the periodic distance to the measured particles.
%
\begin{figure*}
  \includegraphics[width=\textwidth]{Fig1.pdf}
  \caption{
  (a) Demonstration of the $N$ possible centres of mass in an $N$ particle system, where two periodic cells are visible, and the dashed line indicates the particle grouping that leads to a given centre of mass (cross). 
  (b) The problem with the naive centre of mass (red cross) calculation in periodic systems, where the particle group spans a periodic boundary. 
  (c) The projection of the particles in the bottom diagram of (b) onto a circle to find the centre of mass, which minimises the weighted distance to all particles.}
  \label{fig:problems}
  \script{Fig1.py}
\end{figure*}
%

In a non-periodic system, identifying the ``correct'' centre of mass of a group of particles, the average position weighted by the masses, is trivial. 
However, where the periodic boundary can intersect the particle group, this naive calculation may be ``incorrect'' (\cref{fig:problems}b). 
This problem can be resolved where bonding or other relational information is present, but this is not always the case for computational simulation trajectories. 

Bai and Breen proposed an elegant algorithm to efficiently compute the minimised periodic distance centre of mass,\cite{bai_calculating_2008} involving the projection of each orthogonal Cartesian dimension onto a circle to find the centre of mass in the periodic space (\cref{fig:problems}c). 
However, this approach is approximate and introduces numerical error, regardless of whether particles span the periodic boundary. 
Here, we propose an extension to the Bai and Breen algorithm that removes this numerical error, allowing the correct centre of mass to be found in nearly all practical cases. 

\section{Bai and Breen Approach}

To outline the approach of Bai and Breen and the source of the numerical error, we consider a simple one-dimensional periodic system, where the cell ranges from \num{0} to \num{1}. 
This system is analogous to a fractional coordinate system. 
A particle group of $N$ particles can be described with two vectors, one for the positions, $\bm{x}$, and another for the masses, $\bm{m}$. 
Each particle $i$ is then projected onto the two dimensions of a unit circle, $\bm{\xi}$ and $\bm{\zeta}$, where, 
%
\begin{equation}
    \xi_i = \cos(2\pi x_i),
\end{equation}
%
and 
%
\begin{equation}
    \zeta_i = \sin(2\pi x_i).
\end{equation}
% 
The average of the vectors $\bm{\xi}$ and $\bm{\zeta}$ are then found, weighted by the masses;
%
\begin{equation}
    \bar{\xi} = \frac{1}{N}\sum_{i=1}^{N}m_i\xi_i,
\end{equation}
% 
and
%
\begin{equation}
    \bar{\zeta} = \frac{1}{N}\sum_{i=1}^{N}m_i\zeta_i.
\end{equation}
% 
The projection process is then reversed for the weighted average circular coordinates.
Practically, this is achieved using the 2-argument arctangent, 
%
\begin{equation}
    \bar{x} = \frac{\text{atan}2(-\bar{\zeta}, -\bar{\xi}) + \pi}{2\pi},
\end{equation}
%
to give the centre of mass position.
In the case of an orthogonal multi-dimensional system, this method can be repeated independently for each dimension. 

While not discussed in the original work, this method is equivalent to finding the first moment of the Fourier series of the mass density as a function of position (see Appendix \ref{app:math}). 
This approach is used in signal processing to find the spectral centroid of a signal,\cite{tzanetakis_musical_2002} where it is known that it is affected by noise and asymmetry.\cite{teague_robust_2018}
Describing an asymmetric function with a single sine wave leads to an incorrect estimate of the spectral centroid, or centre of mass. 

Considering the limiting case of a 3-particle system, where a central particle is moved between the 2 edge particles.
The error in the centre of mass varies, reaching its maximum very close to the stationary particles, where the system is most asymmetrical (\cref{fig:error_quantification}). 
At the point where the particles are equally distributed, the system is perfectly symmetric, and there is no numerical error in the centre of mass. 
Similarly, there is no error in one or two particle systems as these are inherently symmetrical. 
%
\begin{figure}
    \centering
    \includegraphics[width=8cm]{Fig2.pdf}
    \caption{Quantification of the limiting case (bottom), a 3-particle group spanning almost exactly half the periodic box, where the solid blue line shows the error in the estimate from the Bai and Breen approach normalised by the span (length) of the particle group.
    This error is shown visually for a single example (top), where the Bai and Breen estimate (orange cross) is not close to the true centre of mass (green cross).}
    \label{fig:error_quantification}
    \script{Fig2.py}
\end{figure}
%

\section{\emph{Pseudo}-Centre of Mass Recentering}

We suggest an extension that removes the error introduced by asymmetry, which involves using the Bai and Breen estimate as a \emph{pseudo}-centre of mass. 
The group of particles is repositioned to the middle of the simulation cell using the \emph{pseudo}-centre of mass, such that there is no risk that the group spans a periodic boundary condition. 
Then, the standard weighted mean may be used to find the centre of mass before restoring this to the original space. 

The use of this recentering assumes that the inaccurate Bai and Breen method can find a centre of mass close enough to the true centre of mass that the recentering will place all particles contiguously in the periodic cell. 
This assumption holds true for all systems where the particle group is less than half the size of the periodic cell and is often still true unless the system is very asymmetric. 
If a group of particles spans more than half the size of the periodic box, the only way to guarantee the centre of mass is found accurately is by including bonding or other relational information. 

\section{Method Comparison}

To demonstrate the effectiveness of our \emph{pseudo}-centre of mass recentering extension, \cref{fig:method_comparison} compares it to the Bai and Breen method.
To achieve this, we compute the difference between the estimated centre of mass and the true value as a ratio of the span of the particle group. 
This error is compared with the asymmetry of the particle mass density in the periodic cell.\cite{xioajun_on_1991}
In \cref{fig:method_comparison}, as the asymmetry increases, so does the error in the Bai and Breen approach. 
However, the \emph{pseudo}-centre of mass resampling is not affected in this way. 
Therefore, for most chemical simulations, this \emph{pseudo}-center of mass recentering extension should be used to ensure accuracy in the centre of mass calculation. 
%
\begin{figure}
    \centering
    \includegraphics[width=8cm]{Fig3.pdf}
    \caption{Error in Bai and Breen method (blue shading representing 1, 2, and 3$\sigma$ spreads) compared with the \emph{pseudo}-centre of mass recentering approach (orange line). 
    $2^{24}$ random configurations of one-dimensional periodic boundary-spanning particle groups were generated, with between 3 and 512 particles and a total length of less than half the box. 
    For each configuration, the two approaches were used, and the asymmetry of the particle group\cite{xioajun_on_1991} is compared to the error in the estimate normalised by the span (length) of the particle group. 
    The results were binned along the $x$-axis using 100 bins spaced evenly on a log scale.}
    \label{fig:method_comparison}
    \script{Fig3.py}
\end{figure}
%

\section{Conclusions}

We have presented the \emph{pseudo}-centre of mass recentering extension to the Bai and Breen approach for determining the centre of mass of a group of particles in a periodic cell. 
Unlike the existing approach, this method has no error, even at very large asymmetry values.
To enable the use of this approach, we have implemented it in the open-source Python package \textsc{kinisi}\cite{mccluskey_kinisi_2024} and hope that others will consider its use in future. 

\begin{acknowledgements}
We thank Richard Gowers for the insightful discussion that started us down this rabbit hole and for helpful comments on the manuscript.
\end{acknowledgements}

\section*{Conflict of Interest Statement}

The authors have no conflicts to disclose.

\section*{Author Contributions}
CRediT author statement: 
H.R.: Formal analysis, Investigation, Methodology, Software, Visualisation, Writing – original draft.
J.D.: Conceptualization, Investigation, Methodology, Supervision, Writing – review \& editing.
A.R.M.: Formal analysis, Funding acquisition, Investigation, Methodology, Project administration, Supervision, Writing – review \& editing.

\section*{Data Availability}

The data that support the findings of this study are openly available in a GitHub repository\cite{richardson_github_2025}, under an MIT license, including a complete set of analysis/plotting scripts allowing for a fully reproducible and automated analysis workflow, using \textsc{showyourwork}\cite{luger_showyourwork_2021}.

\appendix

\section{Mathematical Equivalency of Bai and Breen and First Moment Methods}
\label{app:math}

This appendix presents the mathematical equivalency between the Bai and Breen method for calculating the centre of mass, and finding the maximum of the first term of the Fourier series of the mass density as a function of position, known as the first moment method.\cite{teague_robust_2018}

For a real-valued function, the Fourier series can be described as 
%
\begin{equation}
    s_N(x) = a_0 + \sum_{n=1}^{N} \left[ a_n \cos \left( 2\pi \frac{n}{P} x \right) + b_n \sin \left( 2\pi \frac{n}{P} x \right) \right],
    \label{equ:fseries}
\end{equation}
%
where $N$ is some finite number of terms that are computed, and $P$ is the period of the function. 
Eqn.~\ref{equ:fseries} is equivalent to
%
\begin{equation}
    s_N(x) = A_0 + \sum_{n=1}^{N} A_n \cos\left(2\pi \frac{n}{P}x - \varphi_n\right),
\end{equation}
%
by the identity, 
%
\begin{equation}
    \begin{aligned}
        \cos \left( 2\pi \frac{n}{P} x - \varphi_n \right) = & \cos(\varphi_n) \cos \left( 2\pi \frac{n}{P} x \right)  \\ 
        & + \sin(\varphi_n) \sin \left( 2\pi \frac{n}{P} x \right),
    \end{aligned}
\end{equation}
where $\varphi_n$ is the phase shift of the $n$-th harmonic. 
If we write, 
%
\begin{equation}
    a_n = A_n \cos(\varphi_n),
\end{equation}
%
and
%
\begin{equation}
    b_n = A_n \sin(\varphi_n).
\end{equation}
%
Then $a_n$ and $b_n$ can be considered to be constants that ``weight'' the relative contributions of $\sin \left( 2\pi \dfrac{n}{P} x \right)$ and $\cos \left( 2\pi \dfrac{n}{P} x \right)$, akin to the weighted average of $\bm{\xi}$ and $\bm{\zeta}$ carried out within the B\&B method.

As stated in the main text, the Bai and Breen method is equivalent to finding the maxima of the first function of the Fourier series, $f_1$,
%
\begin{equation}
    f_1(x) = a_1 \cos{\Big(2\pi\frac{1}{P} x\Big)} + b_1 \sin{\Big(2\pi\frac{1}{P} x\Big)},
\end{equation} 
%
where $P$ is $1$ as the Bai and Breen method projects the full box size onto a single unit circle. 
To maximise the value of $f_1(x)$ a stationary point is found where,
%
\begin{equation}
    \frac{\partial f_1(x)}{\partial x} = 0,
\end{equation}
%
which is 
%
\begin{equation}
    \frac{\partial f_1}{\partial x} = -2 \pi a_1 \sin( 2\pi x) + 2 \pi b_1 \cos( 2\pi x) = 0.
    \label{equ:der}
\end{equation}
%
Eqn.~\ref{equ:der} can be rearranged to give, 
%
\begin{equation}
\frac{\sin(2 \pi x)}{\cos(2 \pi x)} =  \frac{b_1}{a_1}
\end{equation}
%
and from the trigonometric identity, $\tan(x) = \sin(x) / \cos(x)$, 
%
\begin{equation}
    \tan(2 \pi x) = \frac{b_1}{a_1}.
\end{equation}
%
Therefore the position at which the first function of the Fourier series is maximised, $x$, can be calculated as, 
%
\begin{equation}
    x = \frac{\text{atan}2(b_1, a_1)}{2\pi}.
\end{equation}
%
In terms of periodic position, 
%
\begin{equation}
    \frac{\text{atan}2(b_1, a_1)}{2\pi} \equiv \frac{\text{atan}2(-b_1, -a_1) + \pi}{2\pi}.
\end{equation}
%
The negative $b_1$ and $a_1$ in the Bai and Breen method ensure that the centre of mass is always in the correct periodic box with no additional wrapping.

\bibliography{bib}

\end{document}
